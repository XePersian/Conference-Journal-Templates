\documentclass[a0,12pt,portrait]{a0poster}
\makeatother
\usepackage{multicol} % This is so we can have multiple columns of text side-by-side
\columnsep=100pt % This is the amount of white space between the columns in the poster
%\columnseprule=2.75pt % This is the thickness of the black line between the columns in the poster

\usepackage[svgnames]{xcolor} % Specify colors by their 'svgnames', for a full list of all colors available see here: http://www.latextemplates.com/svgnames-colors

\usepackage{times} % Use the times font
%\usepackage{palatino} % Uncomment to use the Palatino font
%\usepackage{subfig}

\usepackage{graphicx} % Required for including images
\graphicspath{{figures/}} % Location of the graphics files
\usepackage{booktabs} % Top and bottom rules for table
\usepackage[font=small,labelfont=bf]{caption} % Required for specifying captions to tables and figures
\usepackage{amsfonts, amsmath, amsthm, amssymb} % For math fonts, symbols and environments
\usepackage{color,xcolor}
\usepackage[all]{xy}
%\usepackage[margin=1in]{geometry}
%\usepackage[top=20mm, bottom=12mm, left=20mm, right=20mm]{geometry}
\usepackage[top=20mm, bottom=5mm, left=35mm, right=20mm]{geometry}
\usepackage{wrapfig} % Allows wrapping text around tables and figures
\usepackage[pages=some]{background}
\backgroundsetup{
scale=1.02,
color=green,
opacity=0.15,
angle=0,
contents={%
  \centering \includegraphics[width=\paperwidth,height=\paperheight]{poster.jpg}
  }%
}

\usepackage{xepersian}
%\settextfont[Scale=.8]{Yas}
%\setdigitfont[Scale=.7]{Yas}
\settextfont[BoldFont={YasBd.ttf},ItalicFont={YasIt.ttf},BoldItalicFont={YasBdIt.ttf}]{YasBd.ttf}
\setdigitfont{Yas.ttf}

\newtheorem{theorem}{قضیه}[section]
\newtheorem{lemma}[theorem]{لم}
\newtheorem{proposition}[theorem]{گزاره}
\newtheorem{corollary}[theorem]{نتیجه}
\newtheorem{definition}[theorem]{تعریف}
\newtheorem{remark}[theorem]{توجه}
\newtheorem{note}[theorem]{نکته}
\newtheorem{example}[theorem]{مثال}
\renewcommand\proofname{برهان} 
\numberwithin{equation}{section}
%***************************************
%-----------------------------------------------------------
\newcommand{\Q}{\mathbb{Q}}                % for rational number
\newcommand{\R}{\mathbb{R}}                % for Real numbers
\newcommand{\Z}{\mathbb{Z}}                % for Integer number
\newcommand{\nls}{\textbf{n}ls}

\begin{document}
 \BgThispage
 \rightline{{\large\textbf{ چهل و هفتمین کنفرانس ریاضی، دانشگاه خوارزمی}}} \vspace{.5cm}

 \rightline{ \large\textbf{7 تا 10  شهریور 1395}}
\begin{center}
\Huge \textbf{عنوان}\\[0.5cm]  %Title
\huge \textbf{مؤلف اول \footnote{\LARGE مسئول مکاتبات}},
\textbf{مؤلف دوم }\\[0.5cm]
  {\LARGE نام دانشگاه، آدرس ایمیل}
 \\{\LARGE   نام دانشگاه، آدرس ایمیل}\\[0.6cm] 
\end{center}
\vspace{1cm} % A bit of extra whitespace between the header and poster content
\begin{center}
{\huge چکیده }
\end{center}
\huge
     چکیده باید بین 5 تا 10 سطر باشد. فرمولِ شماره‌دار، شکل، جدول و مرجع
      نباید در چکیده آورده شود. چکیده باید {\bf آگاهی دهنده} باشد به‌طوری که کارهای پژوهشی مؤلفین را به‌صورت خلاصه توصیف کند. 
\begin{multicols}{2} % This is how many columns your poster will be broken into, a portrait poster is generally split into 2 columns

\noindent \textbf{واژه‌های کلیدی:} حداکثر 5 واژه یا عبارت

\noindent\textbf{رده‌بندی موضوعی ریاضی \lr{$(2010)$}:}
 به عنوان مثال، \lr{$\mathrm{46J10}$},  \lr{$\mathrm{46J15}$}, \lr{$\mathrm{41A10}$}.
\section{\LARGE مقدمه} 
در اینجا شما باید مقدمات اولیه، اصطلاحات، پیشینه تاریخی، تعریف‌ها و برخی نتایج شناخته شده را بیاورید. 
\begin{definition}
...
\end{definition}
\begin{definition}
...
\end{definition}
\begin{proposition}$\mathrm{[\mbox{\rl{شمارۀ مرجع}}]}$
...
\end{proposition}
\begin{theorem}$\mathrm{[\mbox{\rl{شمارۀ مرجع}}]}$
...
\end{theorem}
\begin{example}

\end{example}
\begin{remark}
...
\end{remark}
\begin{theorem}$\mathrm{[\mbox{\rl{شمارۀ مرجع}}]}$
...
\end{theorem}
\begin{theorem}$\mathrm{[\mbox{\rl{شمارۀ مرجع}}]}$
...
\end{theorem}
\section{\LARGE نتایج اصلی}
   در اینجا باید برخی از مقدمات را ذکر کنید و نتایج اصلی خود را بیاورید، ولی لازم نیست که برهان آنها را هم ذکر کنید. البته می‌توانید برخی از نکات مهم برهان را به‌طور خلاصه ارائه دهید.  ضمناً می‌توانید مثال‌ها و نکته‌هایی را هم بیاورید.  

  به‌جای {\bf{نتایج اصلی}} در عنوان بخش دوم، می‌توانید هر عنوان دیگری را که متناسب با کارهای پژوهشی شما باشد، بنویسید.  ضمناً چنانچه مایل باشید می‌توانید بخش‌های بیشتری را هم اضافه کنید.
\begin{definition}
...
\end{definition}
\begin{proposition}

...
\end{proposition}
\begin{proof}
اگر ترجیح می‌دهید که برهان خلاصه‌ای را ارائه دهید.  
\end{proof}
\begin{theorem}
...
\end{theorem}
\begin{proof}
اگر ترجیح می‌دهید که برهان خلاصه‌ای را ارائه دهید. 
\end{proof}
\begin{example}
...
\end{example}
\noindent{\large{\bf تقدیر و تشکر}}  (اگر مایل هستید)

\noindent مرجع‌ها باید به‌ترتیب الفبایی نام خانوادگی مؤلفین اول باشد. فقط مرجع‌های اصلی آورده شود و تعداد آن‌ها از 7 بیشتر نباشد.


\begin{thebibliography}{99}
\baselineskip=.55cm
\persian
\bibitem{bib1}
 نام و نام خانوادگی مؤلفین، عنوان مقاله، نام مجله (با حروف ایرانیک)، شماره جلد ... (سال انتشار)، صص (از صفحه ... تا ...).
\bibitem{bib2}
نام و نام خانوادگی مؤلفین، عنوان کتاب (با حروف ایرانیک)، ناشر، سال انتشار.
\latin
\bibitem{bib3} H. G. Dales, \textit{Banach Algebras and Automatic Continuity}, Clarendon Press, Oxford, 2000.
\bibitem{bib4} T. J. Ransford, A short proof of Johnson's uniqueness-of-norm theorem, \textit{ Bull. London Math. Soc.} {\bf 21}(1989), 487-488.
\end{thebibliography}
\end{multicols}

\end{document}