\documentclass[a0,12pt,portrait]{a0poster}

\usepackage{multicol} % This is so we can have multiple columns of text side-by-side
\columnsep=100pt % This is the amount of white space between the columns in the poster
%\columnseprule=2.75pt % This is the thickness of the black line between the columns in the poster

\usepackage[svgnames]{xcolor} % Specify colors by their 'svgnames', for a full list of all colors available see here: http://www.latextemplates.com/svgnames-colors

\usepackage{times} % Use the times font
%\usepackage{palatino} % Uncomment to use the Palatino font
%\usepackage{subfig}

\usepackage{graphicx} % Required for including images
\graphicspath{{figures/}} % Location of the graphics files
\usepackage{booktabs} % Top and bottom rules for table
\usepackage[font=small,labelfont=bf]{caption} % Required for specifying captions to tables and figures
\usepackage{amsfonts, amsmath, amsthm, amssymb} % For math fonts, symbols and environments
\usepackage{color,xcolor}
\usepackage[all]{xy}
%\usepackage[margin=1in]{geometry}
%\usepackage[top=25mm, bottom=12mm, left=25mm, right=25mm]{geometry}
\usepackage[top=20mm, bottom=5mm, left=35mm, right=20mm]{geometry}
\usepackage{wrapfig} % Allows wrapping text around tables and figures
\usepackage[pages=some]{background}
\backgroundsetup{
scale=1.02,
color=green,
opacity=0.15,
angle=0,
contents={%
  \centering \includegraphics[width=\paperwidth,height=\paperheight]{poster.jpg}
  }%
}

\newtheorem{theorem}{Theorem}[section]
\newtheorem{lemma}[theorem]{Lemma}
\newtheorem{proposition}[theorem]{Proposition}
\newtheorem{corollary}[theorem]{Corollary}
\theoremstyle{definition}
\newtheorem{definition}[theorem]{Definition}
\newtheorem{example}[theorem]{Example}
\newtheorem{xca}[theorem]{Exercise}
%\theoremstyle{remark}
\newtheorem{remark}[theorem]{Remark}
\numberwithin{equation}{section}
%-----------------------------------------------------------
\newcommand{\Q}{\mathbb{Q}}                % for rational number
\newcommand{\R}{\mathbb{R}}                % for Real numbers
\newcommand{\Z}{\mathbb{Z}}                % for Integer number
\newcommand{\nls}{\textbf{n}ls}

\begin{document}
 \BgThispage
 \leftline{{\Large\textbf{ $\bf 47^{th}$ Annual
Iranian Mathematics Conference}}} \vspace{.5cm}
\leftline{{\Large \textbf{28 -- 31 August 2016}}}
\begin{center}
\Huge \color{NavyBlue} \textbf{Title} \color{Black}\\[0.75cm] % Title
\huge \textbf{Author A\footnote{\Large Corresponding Author}, Author B}\\[0.05cm] % Author(s)
The Name of University, Email Address \\[0.5cm]
The Name of University, Email Address\\[0.5cm] 
\end{center}
\vspace{1cm} % A bit of extra whitespace between the header and poster content

\begin{center}
\Huge Abstract
\end{center}

\huge
The abstract should  contain 5 up to 10 lines. The numbered displayed formulas, figures, tables and references
should not be included in the abstract. The \textbf{abstract should be informative} so that it briefly describes
the research works of the author(s).
\begin{multicols}{2}
\noindent \textbf{Keywords}: At most 5 words or phrases.

\noindent\textbf{Mathematics Subject Classification (2010):} For example, 46J10, 46J15, 41A10.

\section{\LARGE Introduction} 
Here you should bring the preliminaries, terminologies,
 historical background, definitions and some of the known results.

\begin{definition} 
 ...
\end{definition}
\begin{definition}
...
\end{definition}
\begin{proposition}$\mathrm{\cite[Theorem~A]{bib3}}$
...
\end{proposition}
\begin{theorem}$\mathrm{[Reference~ number]}$
...
\end{theorem}
\begin{example}
...
\end{example}
\begin{remark}
...
\end{remark}
\begin{theorem}$\mathrm{[Reference~ number]}$
...
\end{theorem}
\begin{theorem}$\mathrm{[Reference~ number]}$
...
\end{theorem}

\section{\LARGE Main Results}
Here you should present some preliminaries and then bring your main results, but it is not necessary to present the proofs.
 However, you may briefly present some parts of the proofs.
You may also bring some remarks and examples.

\noindent Instead of the Main Results in the title of section 2, you may write any title you wish, which is related to
your research works. You may also add more sections if you prefer.
\begin{definition}
...
\end{definition}
\begin{proposition}
...
\end{proposition}
\begin{proof}
If you prefer to present a short proof.
\end{proof}
\begin{theorem}
...
\end{theorem}
\begin{proof}
If you prefer to present a short proof.
\end{proof}
\begin{example}
...
\end{example}
\noindent{\LARGE\bf Acknowledgement} (If you wish)

\noindent  In the following, references should  appear alphabetically, by the
family names of the first authors.

\begin{thebibliography}{99}
\baselineskip=.5cm
\bibitem{bib1}
 Author's name(s), Title of the paper, \textit{Journal's name}(in Italic)  {\bf Vol. ...} (the year of publication), pages.
\bibitem{bib2}
Author's name(s), \textit{Title of the Book}(in Italic), The
Publisher, the year of publication.
\bibitem{bib3} H. G. Dales, \textit{Banach Algebras and Automatic Continuity}, Clarendon Press, Oxford, 2000.
\bibitem{bib4} T. J. Ransford, A short proof of Johnson's uniqueness-of-norm theorem, \textit{ Bull. London Math. Soc.} {\bf 21}(1989), 487-488.
\end{thebibliography}

\end{multicols}
\end{document}
