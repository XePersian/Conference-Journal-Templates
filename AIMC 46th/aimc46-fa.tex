% !TEX TS-program = XeLaTeX

\documentclass[aimcpersian]{aimc46}
%% در صورت نیاز به فراخوانی بسته‌های اضافی، آن‌ها را در فایل aimc46.cls و قبل از 
%% بسته زی‌پرشین فراخوانی کنید.

\title{
یک نمونه عنوان مقاله آزمایشی
}
\author{
وحید دامن‌افشان%
\RTLthanks{سخنران}\index{دامن‌افشان، وحید}
\university{دانشگاه تبریز}
\and
علی احمدی فضلی
\index{احمدی فضلی، علی}\university{دانشگاه یزد}
%\and
%احمد احمدی
%\index{احمدی، احمد}\university{دانشگاه یزد}
}



\begin{document}
\maketitle

\begin{abstract}
در این قسمت چکیده مقاله در حداقل ۳ و حداکثر ۷
 سطر نوشته می‌شود. در چکیده از نوشتن فرمول نمایشی شماره‌دار، اختصارات غیرمعمول، ارجاع‌دهی به مراجع و امثال آن
 خودداری کنید. به جای آن‌ها، نتیجه اصلی مقاله را به صورت توصیفی بیان کنید.
\end{abstract}
\keywords{توپولوژی اسکات، فضای فشرده، نگاشت بی‌نقص (حداقل ۳ و حداکثر 5 واژه)}
\subject{13D45, 39B42}



\section{مقدمه }
در این بخش، برای نمونه، مقدمه مقاله نوشته می‌شود. مقاله ارسالی باید حداقل ۳ و حداکثر ۴ صفحه داشته باشد و فقط 
در همین قالب نوشته شود.  بدیهی است به مقالاتی که توصیه‌های موجود در قالب را رعایت نکرده باشند، ترتیب اثر داده نخواهد شد. 

این قالب با استفاده از زی‌پرشین تهیه شده است که در هر دو توزیع معروف تک‌لایو و میک‌تک وجود دارد؛ اما پیشنهاد ما استفاده از نسخه‌های
به‌روز توزیع تک‌لایو است. در زی‌پرشین می‌توان هم پانویس فارسی\RTLfootnote{یک پانویس فارسی} داشت و هم پانویس 
انگلیسی\LTRfootnote{An English footnote}. علاوه بر این می‌توان عبارت‌های چندکلمه‌ای انگلیسی را به آسانی و با جهت درست نوشت: 
\lr{Mathematics Conference}.
برای نوشتن بیشتر از چند کلمه به انگلیسی، می‌توان از محیط \texttt{latin} استفاده کرد:
\begin{latin}
 An article is divided into logical units, including an abstract, various sections and subsections, theorems,
 and a bibliography.  The logical units are typed independently of one another.
\end{latin}
سوال‌های فنی خود را درباره این قالب  می‌توانید  با برچسب \lr{aimc46} در سایت پرسش و پاسخ
پارسی‌لاتک%
\LTRfootnote{\texttt{http://qa.parsilatex.com}}
 مطرح کنید. پرسش‌های دیگران را هم می‌توانید در صفحه برچسب 
\lr{aimc46}%
\LTRfootnote{\texttt{http://qa.parsilatex.com/tag/aimc46}}
دنبال کنید.
\begin{definition}
این یک تعریف است که در آن $\sin^2 x+ \cos^2 x=1$ است.
\end{definition}
\begin{theorem}\label{vtheo1}
این یک قضیه است که در آن به مرجع \cite{folland} ارجاع داده می‌شود.

\end{theorem}
\begin{proof}
این یک اثبات است.
\end{proof}
با زدن برچسب‌های مناسب و یکتا به تعاریف، قضایا و... می‌توان در متن به آن‌ها ارجاع داد. به عنوان مثال در اینجا به قضیه \ref{vtheo1}
ارجاع داده می‌شود.
\begin{lemma}
این یک لم است.
\end{lemma}
\begin{proposition}
این یک گزاره است.
\end{proposition}
\begin{corollary}
این یک نتیجه است.
\end{corollary}
\begin{example}\label{vexam1}
این یک مثال است که حل آن در زیر آمده است.
\end{example}
\begin{solution}
این حل مثال \ref{vexam1} است که در آن از \cite{alvarezart} کمک گرفته شده است.
\end{solution}
\begin{remark}
این یک ملاحظه است.
\end{remark}


\section{نتایج اصلی}
در اینجا یک شکل آورده می‌شود. لازم نیست شکل‌ها در همان‌ جایی که در سورس قرار داده می‌شوند، در خروجی هم در همان جا
ظاهر شوند. به جای این کار می‌توان به آن‌ها ارجاع داد. به عنوان مثال شکل  \ref{vfig1} را ببینید.
\begin{figure}[!h]
\centering
\includegraphics[width=5cm]{example-image-a}
\caption{یک شکل آزمایشی}\label{vfig1}
\end{figure}


در اینجا یک جدول آورده می‌شود. لازم نیست جدول‌ها در همان‌ جایی که در سورس قرار داده می‌شوند، در خروجی هم در همان جا
ظاهر شوند. به جای این کار می‌توان به آن‌ها ارجاع داد. به عنوان مثال جدول \ref{vtab1} را ببینید.
\begin{table}[!h]
\centering
\caption{یک جدول آزمایشی}\label{vtab1}
\begin{tabular}{ccc}
\hline
سرستون اول & سرستون دوم & سرستون سوم \\ \hline
نامشخص & $x^2+1$ & $6$ \\ 
$-20$ & $y$ & $11$ \\
$-12$ & $x+y$ & $7$\\
 \hline
\end{tabular} 
\end{table}

همچنین می‌توان با استفاده از بسته \texttt{subfig} دو یا چند شکل را در کنار یکدیگر قرار داد.
\begin{figure}[!ht]
\centering
\subfloat[عنوان شکل اول]{\includegraphics[width=4cm]{example-image-b}}
\quad
\subfloat[عنوان شکل دوم]{\includegraphics[width=4cm]{example-image-c}}
\caption{عنوان کلی شکل}
\label{vfig2}
\end{figure}

حال نوبت به یک فرمول بدون شماره می‌رسد.
\[
\sin^4 x+\cos^4 x=1-2\sin^2 x\cos^2 x.
\]

در ادامه یک فرمول با شماره و با قابلیت ارجاع آورده می‌شود
\begin{align}\label{vequ1}
y=(\sqrt{x}+1)(\sqrt{x}-1)(x+1)
\end{align}
که می‌توان به فرمول \eqref{vequ1} ارجاع داد.

حروف‌چینی فرمول‌های چندخطی نیز ساده است.
\begin{align*}
y&=(\sqrt{x}+1)(\sqrt{x}-1)(x+1)\\
&=(x-1)(x+1)\\
&=x^2 -1.
\end{align*}

همچنین می‌توان فرمول چندخطی، تنها با یک شماره داشت
\begin{align}
y&=(\sqrt{x}+1)(\sqrt{x}-1)(x+1)\notag\\
&=(x-1)(x+1)\label{vequ2}\\
&=x^2 -1\notag
\end{align}
که بعدها بتوان به \eqref{vequ2} ارجاع داد.




\section*{سپاس‌گزاری}
در صورت تمایل، بخش سپاس‌گزاری باید قبل از مراجع نوشته شود.



\begin{thebibliography}{9}
\bibitem{ode}
خیری، حسین، دامن‌افشان، وحید، مقدم، مهسا و وفائی، وجیهه، \textit{نظریه معادلات دیفرانسیل معمولی و سیستم‌های دینامیکی}،
انتشارات دانشگاه تبریز، تبریز، ۱۳۹۰.
\begin{LTRbibitems}
\resetlatinfont
\bibitem{alvarezart}
M. Alvarez-Manilla, A. Jung, K. Keimel, \textit{The probabilistic powerdomain for stably compact
spaces}, Theoretical Computer Science, 328 (2004), pp. 221--244.

\bibitem{alvarezartthe}
M. Alvarez-Manilla, \textit{Measure theoretic results for continuous valuations on partially ordered spaces}, Ph.D. Thesis, Imperial College, University of London, 2001.

\bibitem{folland}
G. B. Folland, \textit{Real Analysis: Modern Techniques and Their Applications}, 2nd ed., John Wiley, 1999.

\bibitem{topsze}
F. Topsze, \textit{Topology and Measure}, Lecture Notes in Mathematics, Vol. 133, Springer, Berlin, 1970.
\end{LTRbibitems}
\end{thebibliography}

\email{vdamanafshan@gmail.com}
\email{author2@aaa.ac.ir}
%\email{author3@aaa.ac.ir}
\end{document}